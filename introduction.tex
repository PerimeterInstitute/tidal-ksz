\indent
Over past decades, a deficiency of baryons in local universe has been noticed comparing to the
baryon density of the early universe , which was fixed by the cosmic microwave background,Big Bang Nucleosynthesis and Lyman-$\alpha$ forest\cite{Cook14}\cite{Fukugita98}\cite{Komatsu11}\cite{Hinshaw13}.
At $z\lessim 2$ the observed baryons in collapsed
objects such as galaxies, galaxy clusters and groups only account for ~10$\%$ of the predicted amount.
More baryons are believe to reside in Warm-Hot Intergalactic Mediums (WHIM) with typical temperature of $10^5$ K to $10^7$ K\cite{Soltan06}.
Due to low density and relatively low temperature, it is difficult to measure.
Over the years, continuous effort has been made to detect this part of the baryons, using absorption lines of hydrogen and metal in different ionizations states(eg, HI, Mg II,Si II, C II, Si III, C III, Si IV, O VI, O VII).\cite{Salucci99}\cite{Werk}
However, these detections are usually limited to circumgalactic medium, while at least 25\% of the baryons are in more diffused region\cite{Dave2010}. Moreover, the uncertainty in metalicity would reduce the reliability.

A possibly more promising tool to probe the missing baryon may be the kinetic Sunyaev-Zel'dovich(kSZ) effect\cite{Sunyae72}\cite{Sunyaev80}. 
It refers to the secondary temperature anisotropy in CMB caused by radial motions of free electrons.
It is promising because: first, it gives an integrated signal coming from all the electrons, 
so we should in principle be able to find signals of all baryons as long as they have ionized electrons; 
second, unlike thermal SZ effect, kSZ signal is only related to the density and velocity of the electrons, and is less confined to collapesed objects with higher thermal temperature and electron pressure. 
Moreover, since the prominent contribution to the velocity field comes from large scale structures, it is less biased towards local high density and therefore serve as better probe for total baryon distribution.

Due to the contamination from primary CMB and residual thermal SZ signal, as well as the loss of redshift information, it is not easy to measure ksz signal by itself. 
Previous approches tend to cross relate the CMB temperature anisotropy with galaxy surveys. 
whether using pairwise-momentum estimator\cite{Hand2012} or velocity-field-reconstruction estimator\cite{Shao2011}\cite{Li2014}, 
a common requirement is that they need spectroscopy of galaxies to provide accurate redshift. 
And this largely limit the volume and redshift range to apply the method. 
A fairly recent effort try to fix this by using photometries of infrared-selected galaxies, 
and cross relate them with squared ksz signal. 
However, since they used projected fields of the galaxies, they lost the information of individual redshift\cite{Hill2016}.

In this paper, we present a new cross relating source---neutral hydrogen density field, 
that may help probe the baryon content to redshift one in very near future.
The field can be obtained from 21cm intensity mapping surveys, a kind of survey that integrate spectrum of diffuse line radiation rather than detect individual objects. 
It is designed to detect weak, diffuse HI signals, and therefore can be easily extended to probe higher redshift universe.
Moreover, the 21cm spectrum contains relatively accurate redshift information, which makes it a good candidate to cross relate with kSZ signals.
However, it also has a great drawback. The continuum foregrounds of 21cm measurement could be $10^2 - 10^3$ times brighter than cosmological signal, 
almost completely bury the information about large scale density field in radial direction, i.e. modes with small $k_\parallel$. 
On the other hand, the large scale structure make contributions to the distribution of velocity, which make the corelation seem nearly impossible previously.

However, a new method called {\it cosmic tidal reconstruction} has been 
developed recently\cite{2012:pen}\cite{2015:zhu}, 
it can reconstruct the large scale density field from the alignment of small 
scale cosmic structures. Applying it, we will be able to cross correlate the kSZ signal with 21cm density field.

The paper is organized as follows. In section II, we introduce the tidal reconstruction methods, and how to use the reconstructed density field to correlate with the kSZ signal.
In section III, we address the simulation setup and results. 
In section IV, we discuss the error scale and future applications.
