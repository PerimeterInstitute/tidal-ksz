\begin{abstract}
	We propose a new way to study baryon abundance and distribution in local universe.
	We cross correlate density field from HI 21cm intensity mapping with temperature anisotropy of Cosmic Microwave Background(CMB) caused by raidal motion of free electrons, eg. kinetic Sunyaev-Zel'dovich (kSZ) effect. 
	We apply cosmic tidal reconstruction to recover the modes lost in 21cm foreground subtraction, and this effectively promote the apprearance of correlation signal.

	The method is less biased towards local density contraction, 
	Since kinemic motion is mainly related to large scale structures.
	Precise redshift information can be obtained from 21cm spectrum. 
	The method could be easily applied to redshift much higher than current spectroscopic galaxy surveys.

	We tested the idea using simulations performed on $z=1$, with realistic foreground noise level.
	We successfully recover $60\%\sim70\%$ signal of the 21cm density field at $k\sim 0.01 h/Mpc$ after tidal reconstruction. 
	We correlate it with kSZ signals using velocity-reconstruction method.
	We obtain a $r\sim0.3$ correlation from $l\sim100-2000$, which indicates a detectable signal.

	Consider the large volume and high depth of current and future 21cm surveys,
	our proposal will compensate for the study on relating kSZ signal with spectroscopic or even photometric galaxy surveys.
\end{abstract}


