\begin{abstract}
21cm intensity mapping has emerged as a promising technique to map the 
large scale structure of the Universe, at redshifts $z$ from 1 to 10.
Unfortunately, many of the key cross correlations with photo-$z$ galaxies
and the CMB have been thought to be impossible due to foreground
contamination for radial modes with small wavenumbers \tcb{(copied)}. 
These modes are usually subtracted in the foreground subtraction process.
We recover lost 21cm radial modes via cosmic tidal reconstruction and 
find more than 60\% cross correlation signal at $\ell\lesssim100$ and 
even more on larger scales can be recovered from null. 
The tidal reconstruction method opens up a new set of possiblities to 
probe our Universe and is extremely valuable not only for 21cm surveys
but also CMB and photometric redshift observations.
\end{abstract}


