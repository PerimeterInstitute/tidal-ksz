\indent
The temperature fluctuations caused by kSZ effect is:
\begin{eqnarray}
\label{eq:ksz}
\Theta_{kSZ}(\hat n)\equiv\frac{\Delta T_{kSZ}}{T_{CMB}}
=-\frac{1}{c}\int d\eta  g(\eta)  \bm{p}_\parallel\ ,
\end{eqnarray}
where $\eta(z)$ is the comoving distance at redshift z, $g(\eta)=e^{-\tau} d\tau/d\eta$ , $\tau$ is the optical depth to Thomson scattering, $\bm{p}_\parallel=(1+\delta)\bm{v}_\parallel$, with $\delta$ the electron overdensity. 

Since direct cross correlations between kSZ signal and density field will vanish due to the cancellation of positive and negative velocity, we first construct a 3D velocity field\cite{Shao11} from the clean large scale density field $\hat\kappa_c$.

In linear region, the continuity equation goes like:
$\dot \delta+\nabla \cdot \bm{v}=0$, 
where $\bm{v}$ is the peculiar velocity and $\delta$ is the matter overdensity. 
Since the 21cm density field is believed to well trace the dark matter fields in low redshift, we use $\kappa_c$ here, instead of $\delta$.

Therefore, we get velocity field:
\begin{eqnarray}
	\label{eq:v}
\hat v_z(\bm{k})=i a H \frac{d\mathrm{ln}D}{d\mathrm{ln}a}\kappa(\bm{k})\frac{k_z}{k^2}\,
\end{eqnarray}
where D(a) is the linear growth function.

As we can see, $v_z \propto \frac{k_z}{k^2}$, indicating the most prominent signal comes from small k mode, which corresponds to large scale structure. This partly consolidate our motivation for the tidal reconstruction procedure.

To compare with the original kSZ signal, we calculate the $\hat\Theta(\bm{n})$ using reconstructed velocity field. 

However, before apply Eq.(\ref{eq:ksz}) we notice that the additional term $\frac{k_z}{k^2}$ in Eq.(\ref{eq:v})
will strongly amplify the noise in small k modes. Therefore, we apply a Wiener filter similar to Eq.(\ref{eq:wiener}) for the velocity field.

Then we use the orignal 21cm density field as $\delta$, 
and reconstruct the kSZ signal following Eq.(\ref{eq:ksz}). 
We compare it with the original signal directly from simulations.


