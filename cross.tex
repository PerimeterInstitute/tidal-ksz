Due to the cancellation of positive and negative velocity, direct cross correlation between kSZ signal and density field will vanish.
Therefore, we first estimate the peculiar velocity from the 3D density field, 
then construct the 2D map of kSZ signal, 
finally correlate it with the real kSZ signal \cite{Shao11}.

Detailed steps are as follows.

(1) Estimate the velocity field.

In linear region, the continuity equation goes like:
$\dot \delta+\nabla \cdot \bm{v}=0$, 
where $\bm{v}$ is the peculiar velocity and $\delta$ is the matter overdensity. 

Therefore, we obtain an estimator of velocity distribution:
\begin{eqnarray}
	\label{eq:v}
\hat v_z(\bm{k})=i a H \frac{d\mathrm{ln}D}{d\mathrm{ln}a}\delta(\bm{k})\frac{k_z}{k^2}\,
\end{eqnarray}
where D(a) is the linear growth function.

As we can see, $v_z \propto \frac{k_z}{k^2}$, indicating the most prominent signal comes from small k mode, which corresponds to large scale structure. 
This further verify our motivation for tidal reconstruction procedure.

(2) suppress the noise in velocity field with a new Wiener filter.

The additional term $\frac{k_z}{k^2}$ in Eq.(\ref{eq:v})
will strongly amplify the noise in small k modes. 
Therefore, we apply a Wiener filter similar to Eq.(\ref{eq:wiener}) for the velocity field.
\tcp{to be continued}

(3) calculate the 2D kSZ map.

The CMB temperature fluctuations caused by kSZ effect is:
\begin{eqnarray}
\label{eq:ksz}
\Theta_{kSZ}(\hat n)\equiv\frac{\Delta T_{kSZ}}{T_{\mr{CMB}}}
=-\frac{1}{c}\int d\eta  g(\eta)  \bm{p}_\parallel\ ,
\end{eqnarray}
where $\eta(z)$ is the comoving distance at redshift z, $g(\eta)=e^{-\tau} d\tau/d\eta$ is the visibility function, $\tau$ is the optical depth to Thomson scattering, $\bm{p}_\parallel=(1+\delta)\bm{v}_\parallel$, with $\delta$ the electron overdensity. 

We assume that $g(\eta)$ doesn't change significally in one redshift bin, 
and integrate $\bm{p}_\parallel$ along radial axis to get $\hat \Theta_{kSZ}$
%Then we use the orignal 21cm density field as $\delta$, 
%and reconstruct the kSZ signal following Eq.(\ref{eq:ksz}). 
%We compare it with the original signal directly from simulations.


